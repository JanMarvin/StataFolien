%%%
%%% Statistics
%%%
\part{Statistics}
\begin{frame}
\thispagestyle{empty}
\textbf{\huge{Statistics}}
\end{frame}

\begin{frame}{Statistics Contents}
 \tableofcontents
\end{frame}


\section{Statistic}
\begin{frame}[fragile]{Statistic I}
Arithmetic mean and some additional statistics \index{Mean!mean} \index{Describe!summarize} \index{Describe!detail} \index{Mean} \index{Statistic}
\begin{lstlisting}
  mean height
  summarize height
  summarize height, detail
\end{lstlisting}

  \begin{tikzpicture}[transform shape, rotate=10, overlay]
\node at (8,-1.5) [mybox] (box) {%
    \begin{minipage}[t!]{0.35\textwidth}
    \tiny\textcolor{black}{\texttt{mean returns the standard error: $s/\sqrt{n}$. summarize returns the standard deviation: $\sqrt{\sum(x-\bar{x})^2)/n}$.}}
    \end{minipage}
    };
\end{tikzpicture}
\end{frame}


\begin{frame}[fragile]{Statistic II}
Minimum, maximum, arithmetic mean, median, number of observations and quartils. \index{Minimum} \index{Minimum!min} \index{Maximum} \index{Maximum!max} \index{Mean!mean} \index{Mean} \index{Mean!arithmetic mean} \index{Mean!Median} \index{Quantile} \index{tabstat} \index{Range} \index{Range!range} \index{Quantile!q}
\begin{lstlisting}
  tabstat height, statistic(min max mean median p50)
  tabstat height, statistic(min max range mean count q) by(county)
\end{lstlisting}

  \begin{tikzpicture}[transform shape, rotate=10, overlay]
\node at (8,-1.5) [mybox] (box) {%
    \begin{minipage}[t!]{0.35\textwidth}
    \tiny\textcolor{black}{\texttt{range = max - min}}
    \end{minipage}
    };
\end{tikzpicture}
\end{frame}


\begin{frame}[fragile]{Statistics III}
Standard deviation, standard error, variance und interquartil range. \index{Standard deviation} \index{Standard error} \index{Variance} \index{Interquartil range} \index{tabstat} \index{Standard deviation!sd} \index{Variance!var} \index{Standard error!sem} \index{Interquartil range!iqr} \index{Quantile!q} \index{Skewness} \index{Kurtosis} \index{Skewness!skewness} \index{Kurtosis!kurtosis}
\begin{lstlisting}
 tabstat height, statistic(sd sem var q iqr)
\end{lstlisting}
Skewness and kurtosis
\begin{lstlisting}
  tabstat height, statistics(skewness kurtosis)
\end{lstlisting}

  \begin{tikzpicture}[transform shape, rotate=10, overlay]
\node at (8,-1.5) [mybox] (box) {%
    \begin{minipage}[t!]{0.35\textwidth}
    \tiny\textcolor{black}{\texttt{iqr = q3 - q1}}
    \end{minipage}
    };
\end{tikzpicture}
\end{frame}

\section{Tables}
\begin{frame}[fragile]{Tables} \index{Table!tab} \index{Table!tab, summarize()} \index{Table}
We already had \texttt{summarize} and \texttt{tab}, let us combine them
\begin{lstlisting}
  tab county, summarize(height)
\end{lstlisting}
\end{frame}

\subsection{Cross tabs}
\begin{frame}[fragile]{Cross tabs} \index{Table!tab} \index{Table!tab1} \index{Table!tab2} \index{Table!cross tab}
Create a cross tab with
\begin{lstlisting}
  tab school county
\end{lstlisting}
And some more tables
\begin{lstlisting}
  ** tabs of each variable
  tab1 sex county school
  ** cross tabs of ab bc ac
  tab2 sex county school
\end{lstlisting}
\end{frame}

\subsection{Chi, V and Phi}
\begin{frame}[fragile]{Cross tabs II} \index{Table!cross tab} \index{Table!$\chi^2$} \index{Table!chi-square} \index{Table!chi} \index{Table!tab} \index{Table!Cramers V} \index{Table!tab exp} \index{Table!tab col} \index{Table!tab row}
Let us recapitulate Statistics I, we will do some tests \footnote{or see e.g. \textcite{Agresti09}.}
\begin{lstlisting}
** chi^2
tab sex county, chi
** cramers v
tab sex county, V

** manual
tab sex county, exp col row
** phi and for 2x2 tabs V
di (1013*1002 - 925*1131) / (2144*1927*1938*2133)^(1/2) 
** chi^2
di 4071 *(-.00753735)^2
** usual way to compute V
di (.23128021 / 4071 * (2 - 1) )^(1/2) // V
\end{lstlisting}
\end{frame}

\begin{frame}[fragile]{Numlabel}
You can add numeric values to your variable labels \index{Label!numlabel}
\begin{lstlisting}
  ** numeric labels on for all variables
  numlabel _all, add
  ** and off
  numlabel _all, remove
\end{lstlisting}
\end{frame}

\section{Inference}
\begin{frame}[fragile]{T-Test} \index{T-Test!ttest} \index{T-Test}
And finally let us have a look at some test-statistics
\begin{lstlisting}
  ** compare 
  tab sex, sum(bmi)
  ttest bmi, by(sex)
\end{lstlisting}
\end{frame}

\begin{frame}[fragile]{And much more to come \dots} \index{Regression!regress} \index{Regression!logit} \index{Regression!mlog}
Regression
\begin{lstlisting}
  regress
  logit
  mlog
\end{lstlisting}
\end{frame}

\begin{frame}
\thispagestyle{empty}
 To be continued.
\end{frame}
