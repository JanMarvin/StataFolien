%%%
%%% Variablen
%%%
\part{Variablen}
\begin{frame}
\thispagestyle{empty}
\textbf{\huge{Variablen}}
\end{frame}

\begin{frame}{Variablen Inhalt}
 \tableofcontents
\end{frame}

\section{Grundlagen}
\subsection{Generieren}
\begin{frame}[fragile]{Generieren} \index{Generieren!generate} \index{Generieren!gen} \index{Generieren!clonevar} \index{Rekodieren!recode} \index{Tabelle!tab} \index{Generieren}
Hin und wieder ist es notwendig Variablen zu erstellen.
\begin{lstlisting}
  ** Variable klonen
  clonevar sex = v298
  ** Variable generieren
  generate geschlecht = .
  recode geschlecht . = 1 if sex==1
  recode geschlecht . = 2 if sex==2

  tab sex
  tab geschlecht
\end{lstlisting}
\end{frame}

\subsection{Umbenennen}
\begin{frame}[fragile]{Umbenennen}
Nach einigem hin und her gefällt uns der Variablenname \texttt{geschlecht} nicht mehr. \index{Umbenennen!rename} \index{Umbenennen}
\begin{lstlisting}
 drop sex // die Variable werfen wir weg 
 ** geschlecht heißt jetzt sex
 rename geschlecht sex
\end{lstlisting}
\end{frame}

\begin{frame}[fragile]{Moment \texttt{generate \dots~= .} ?}
Durch den Punkt, wird in Stata ein fehlender Wert gekennzeichnet. \index{Missing Values!.} \index{Rekodieren!recode}
\begin{lstlisting}
  recode sex 9 = .
  ** Verschiedene fehlende Werte
  recode sex 8 = .a
  recode sex 7 = .b
  ...
  recode sex -3 = .z
\end{lstlisting}

  
  \begin{tikzpicture}[transform shape, rotate=10, overlay]
\node at (7.5,-1.5) [mybox] (box) {%
    \begin{minipage}[t!]{0.35\textwidth}
    \tiny\textcolor{black}{\texttt{Der Punkt dient darüberhinaus auch als Zeichen für $+\infty$ und ist die größte Stata bekannte Zahl.}}
    \end{minipage}
    };
\end{tikzpicture}

\end{frame}

\begin{frame}[fragile]{Moment \texttt{\dots~if \dots~== 1} ?}
% Durch \texttt{if} werden in Stata Bedingungen festgelegt. Im Beispiel wird die Variable \texttt{geschlecht} auf den Wert $1$ gesetzt, wenn die Variable \texttt{sex} den Wert $1$ hat. Solche Bedingungen sind für eine ganze Reihe von Befehlen möglich und können im Regelfall auch verknüpft werden.
Bedingungen \index{Tabelle!tab} \index{if} \index{if!und \&} \index{if!oder \|}
\begin{lstlisting}
** Haushaltseink ostdeutscher Haushalte mit Bildungsabschluss 1 oder 2
  tab hhinc if east == 1 & ///
    (education == 1 | education == 2)
\end{lstlisting}
  \begin{tikzpicture}[transform shape, rotate=10, overlay]
\node at (7.5,-1.5) [mybox] (box) {%
    \begin{minipage}[t!]{0.35\textwidth}
    \tiny\textcolor{black}{\texttt{Nach if kommt die Bedingung, verschiedene Bedingungen können mit $\&$ und $|$ verknüpft werden. Bei Verkettungen muss trotzdem immer der Variablenname angegeben werden.}}
    \end{minipage}
    };
\end{tikzpicture}
\end{frame}

\begin{frame}[fragile]{Moment \texttt{tab} ?}
% Der Befehl \texttt{tab} ist eine Kurzform von tabulate. Kurzschreibweisen findet man durch das Studium der Helpfiles.\vspace{0.5cm}
Kurzschreibweisen\\ \vspace{0.5cm} \index{Tabelle!tab} \index{Hilfe!help}
\begin{scriptsize}
\texttt{
  \textcolor{Statakeywords}{help tabulate twoway} \\
  ~\underline{tab}ulate \textcolor{Statakeywords}{varname1} \textcolor{Statakeywords}{varname2} [\textcolor{Statakeywords}{if}] [ \dots
} \vspace{0.5cm}
\end{scriptsize}

  \begin{tikzpicture}[transform shape, rotate=10, overlay]
\node at (7.5,-1.5) [mybox] (box) {%
    \begin{minipage}[t!]{0.35\textwidth}
    \tiny\textcolor{black}{\texttt{Der unterstrichene Teil steht für die minimal notwendige Anzahl an Buchstaben, die gebraucht werden, damit Stata den Befehl eindeutig erkennt.}}
    \end{minipage}
    };
\end{tikzpicture}
\end{frame}

\subsection{Labeln}
\begin{frame}[fragile]{Label} \index{Label!label} \index{Label!label variable} \index{Label!label define} \index{Label!label values} \index{Tabelle!nolabel}
Label erst definieren, dann zuweisen
\begin{lstlisting}
  ** Variable labeln
  label variable geschlecht "Geschlecht"
  ** Label für Ausprägungen definieren
  label define labgeschlecht 1 "Männlich" 2 "Weiblich"
  ** Label für Ausprägungen der Variable zuschreiben
  label values geschlecht labgeschlecht 
\end{lstlisting}
Jetzt können die Label angezeigt werden
\begin{lstlisting}
  tab geschlecht
  tab geschlecht, nolabel
\end{lstlisting}
\end{frame}

\section{Fortgeschritten}
\subsection{Generieren mit Mathe}
\begin{frame}[fragile]{Generieren II} \index{Generieren!gen} \index{Generieren}
Variablen können auch durch mathematische Operationen erzeugt werden.
\begin{lstlisting}
  ** Neu Variable "alter" generieren
  gen alter = 2011-gebjahr
  sum alter

  ** Haushaltseinkommen
  gen hhinc = inc_female + inc_male
  ** Haushaltsäquivalenzeinkommen
  gen equiv = hhinc * equiscale
\end{lstlisting}
\end{frame}

\subsection{Rekodieren in Variable}
\begin{frame}[fragile]{Generieren III} \index{Generieren!gen} \index{Rekodieren!recode}
Eine neue Variable wird aus rekodierten Werten erzeugt, dabei bleibt die Originalvariable unberührt.
\begin{lstlisting}
  recode var1 ///
    (-1=.) ///
    (1=1 "Vollzeitbeschäftigung") ///
    (2=2 "Teilzeitbeschäftigung") ///
    (3 5 6 7 8 = 99 "Sonstige Beschäftigung") ///
    (4=3 "geringfügige Beschäftigung") ///
    (9=5 "Keine Beschäftigung"), gen (Beschaeftigung)
\end{lstlisting}

  \begin{tikzpicture}[transform shape, rotate=10, overlay]
\node at (7.5,-1.5) [mybox] (box) {%
    \begin{minipage}[t!]{0.35\textwidth}
    \tiny\textcolor{black}{\texttt{Variablennamen ohne Sonderzeichen und mit Buchstabe am Anfang.}}
    \end{minipage}
    };
\end{tikzpicture}

\end{frame}

\begin{frame}[fragile]{Moment \texttt{///} ?} \index{display} \index{Zeilenumbruch} \index{Zeilenumbruch!///} \index{Zeilenumbruch!delimit}
Die drei aufeinanderfolgenden \textit{Slashs} sagen Stata, dass der Befehl sich über die nächste Zeile erstreckt. 
\begin{lstlisting}
  ** alles in einer Zeile
  display 1 + 1 
  ** jetzt mit (unnötigem) Zeilenumbruch
  display 1 + ///
    1
\end{lstlisting}

  \begin{tikzpicture}[transform shape, rotate=10, overlay]
\node at (7.5,-1.5) [mybox] (box) {%
    \begin{minipage}[t!]{0.35\textwidth}
    \tiny\textcolor{black}{\texttt{So können zwei und mehr Zeilen verbunden werden. Dadurch wird die Syntax lesbarer. Alternativ: help delimit}}
    \end{minipage}
    };
\end{tikzpicture}
\end{frame}

\subsection{Klassifizieren}
\begin{frame}[fragile]{Rekodieren} \index{Klassifizieren} \index{Generieren!clonevar} \index{Rekodieren!recode}
Manchmal bietet es sich an metrische Variablen zu klassifizieren:
\begin{lstlisting}
  clonevar alter_kl = alter
  recode alter_kl ///
    (18/20=1) (21/30=2) ///
    (31/40=3) (41/50=4) ///
    (51/60=5) (61/110=6)
  tab alter_kl
\end{lstlisting}

  \begin{tikzpicture}[transform shape, rotate=10, overlay]
\node at (7.5,-1.5) [mybox] (box) {%
    \begin{minipage}[t!]{0.35\textwidth}
    \tiny\textcolor{black}{\texttt{18/20 meint hier von 18 bis 20. 21/30 von 21 bis 30. Achten Sie auf die Klassenbreite!}}
    \end{minipage}
    };
\end{tikzpicture}
\end{frame}

\subsection{Fehlende Werte}
\begin{frame}[fragile]{Daten aufbereiten} \index{Missing Values} \index{Missing Values!mvdecode} \index{Missing Values!mv()}
Ihr Datensatz erhält, für das Einkommen Angaben von $-3$, $-2$ und $-1$. Dem Codebuch entnehmen Sie, dass diese für \glqq keine Angabe\grqq, \glqq Antwort verweigert\grqq~und \glqq weiß nicht\grqq~stehen.
Wenn Sie sich den Mittelwert über das Einkommen angeben lassen, wird dieser -- durch die Angaben die kleiner $0$ sind -- verzehrt.
\begin{lstlisting}
  mean hhinc
\end{lstlisting}
Deshalb kodieren Sie die fehlenden Werte als Missings.
\begin{lstlisting}
  mvdecode hhinc, mv(-3=.c\-2=.b\-1=.a)
  mean hhinc
\end{lstlisting}
\end{frame}

\subsection{Variablen Handhabung}
\begin{frame}[fragile]{Fehler} \index{Fehler} \index{drop}
% \begin{minipage}{11cm}
Sie haben Variablen erstellt und kodiert. Sie möchten die soeben erstelle Variable mit den klassifizieren Altern noch einmal anpassen. Was ist zu tun?
\begin{itemize}
 \item Glücklicherweise haben Sie die Originalvariablen nicht angerührt. Sie können nun einfach die Syntax nochmals von Anfang an durchlaufen lassen.
 \item Sie kennen die Stelle mit dem Fehler und wollen die falsch erstellte Variable löschen
 \begin{lstlisting}
  drop alter_kl
 \end{lstlisting}
\end{itemize}
% \end{minipage}
\end{frame}

\section{Speichern und Laden}
\subsection{Speichern}
\begin{frame}[fragile]{Daten sichern} \index{Speichern!save} \index{Speichern!replace} \index{Speichern}
% Ein langer erfolgreicher Arbeitstag geht zuende. Sie wollen Ihre bisherigen Ergebnisse sichern.
Speichern Sie ihr Do-File in einen Ordner \textit{do-files} und ihre Daten in einen Ordner \textit{data}.
\begin{lstlisting}
  save "D:/Daten/data/testdaten_rekodiert.dta"
\end{lstlisting}
Zum erneuten Speichern muss ein anderer Dateiname angegeben werden oder die Option \texttt{replace} genutzt werden.
\begin{lstlisting}
  save "D:/Daten/data/testdaten_rekodiert.dta", replace
\end{lstlisting}

  \begin{tikzpicture}[transform shape, rotate=10, overlay]
\node at (7.5,-1.8) [mybox] (box) {%
    \begin{minipage}[t!]{0.35\textwidth}
    \tiny\textcolor{black}{\texttt{Einmal ersetzte Datensätze sind unwiderruflich überschrieben. Überschreiben Sie \underline{niemals} die Originaldaten!}}
    \end{minipage}
    };
\end{tikzpicture}
\end{frame}

\subsection{Laden}
\begin{frame}[fragile]{Daten wieder einlesen} \index{Laden!use} \index{Laden!clear} \index{Laden}
Mit der Option \texttt{clear} wird der aktuelle Datensatz ersetzt.
\begin{lstlisting}
  use "D:/Daten/data/testdaten.dta", clear
\end{lstlisting}

  \begin{tikzpicture}[transform shape, rotate=10, overlay]
\node at (7.5,-1.8) [mybox] (box) {%
    \begin{minipage}[t!]{0.35\textwidth}
    \tiny\textcolor{black}{\texttt{Mögliche Veränderungen an vorhandenen eingelesenen Datensätzen werden ignoriert. Sie zwingen Stata hier ein Verhalten auf, vor dem Stata Sie sonst normal warnt.}}
    \end{minipage}
    };
\end{tikzpicture}
\end{frame}