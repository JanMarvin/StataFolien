%%%
%%% Einstieg
%%% 
\part{Einstieg}
\begin{frame}
\thispagestyle{empty}
\textbf{\huge{Einstieg}}
\end{frame}

\begin{frame}{Einstieg Inhalt}
 \tableofcontents
\end{frame}


\section{Datentypen}
\subsection{Syntaxdateien}
\begin{frame}[fragile]{Syntax}
 Wir schreiben Syntax.
 \begin{itemize}
  \item Reproduzierbar!
  \item Weniger \dots
  \begin{itemize}
   \item klicken
   \item wiederholen
   \item fehleranfällig
  \end{itemize}
 \end{itemize}
\end{frame}

\begin{frame}[fragile]{Syntaxdateien}
  \begin{itemize}
  \item Sytax in Stata in do-Dateien. \index{do-Files}
  \item Aufruf des Editors \index{do-Files!doedit}
  \begin{lstlisting}
  doedit
  \end{lstlisting}
  \item Dateiendung von do-Dateien ist .do.
  \item do-Dateien sind mit jedem Texteditor zu öffnen und zu bearbeiten.
  \end{itemize}
\end{frame}

\subsection{Datendateien}
\begin{frame}{Sekundärdaten}
 \begin{itemize}
  \item Primärdaten
  \item Sekundärdaten
 \end{itemize}

 \begin{tikzpicture}[transform shape, rotate=10, overlay]
\node at (7.5,-1.5) [mybox] (box) {%
    \begin{minipage}[t!]{0.35\textwidth}
    \tiny\textcolor{black}{\texttt{Wir nutzen Daten. Wir erheben keine Daten. Wir erstellen und editieren keine Daten.}}
    \end{minipage}
    };
\end{tikzpicture}

\end{frame}

\begin{frame}{dta-Dateien}
\index{dta-Dateien}
  \begin{itemize}
  \item Daten werden als dta-Dateien gespeichert
  \item Daten enden auf .dta
  \item Stata Datenobjekte können von Stata und einigen Statistik-Programmen z. B. R (\cite{R13}) gelesen werden.
  \end{itemize}
\end{frame}

\begin{frame}[fragile]{Daten einlesen}
  \begin{itemize}
    \item Verzeichnis anzeigen \index{Verzeichnis!dir}
    \begin{lstlisting}
    dir
    \end{lstlisting}
    \item Verzeichnis wechseln \index{Verzeichnis!cd}
    \begin{lstlisting}
    cd "D:/Daten/data/"
    \end{lstlisting}
    \item Daten laden \index{use} \index{Laden}
    \begin{lstlisting}
    use <dateiname.dta>
    \end{lstlisting}
  \end{itemize}
  \index{Umlaute}
  \index{Leerzeichen}
  
   \begin{tikzpicture}[transform shape, rotate=10, overlay]
\node at (7.5,-0.5) [mybox] (box) {%
    \begin{minipage}[t!]{0.35\textwidth}
    \tiny\textcolor{black}{\texttt{Vermeiden Sie Ordner- und Dateinamen mit Umlauten oder Leerzeichen. Stata mag das nicht.}}
    \end{minipage}
    };
\end{tikzpicture}
  
\end{frame}

\section{do-Files}
\begin{frame}[fragile]{do-File I}
  \begin{itemize}
  \item Dateikopf: Dateiname, der Name des Erstellers, das Erstellungsdatum, eine kurze Information, was die Datei macht und bei längeren Dateien ein kurzes Inhaltsverzeichnis. 
  \index{Dateikopf}
  \index{do-Files}
  \index{Kommentar}
    
  \begin{lstlisting}
  /*
  Name: read-soep.do
  Autor: Garbuszus
  Datum: 2013-02-28
  Inhalt: Einlesen der SOEP-Daten
  */
  \end{lstlisting}
  \end{itemize}
\end{frame}

\begin{frame}{do-File Ia}
Dokumentieren Sie! \index{Kommentar}
\begin{itemize}
\item Teamarbeit
\item Vergesslichkeit
\end{itemize}

  
     \begin{tikzpicture}[transform shape, rotate=10, overlay]
\node at (7.5,-0.5) [mybox] (box) {%
    \begin{minipage}[t!]{0.35\textwidth}
    \tiny\textcolor{black}{\texttt{Niemand möchte Ihre Syntax studieren, dokumentieren Sie. Dokumentieren Sie viel und gründlich, Sie werden vergessen!}}
    \end{minipage}
    };
\end{tikzpicture}

\end{frame}


\begin{frame}[fragile]{do-File Ib}
\begin{itemize} \index{Kommentar!kommentieren}
   \item \texttt{/*} (Anfang) und \texttt{*/} (Ende) definieren einen Kommentar. Alternativ geht das auch zu Beginn einer Zeile mit \texttt{*} oder nach einem Befehl mit \texttt{//}
  
  \begin{lstlisting}
  ** Wir schreiben den Dozenten zuliebe Code fuer Version 12
  version 12 // Wir haben zwar Stata 13, die aber nicht.
  \end{lstlisting}
  \item Befehle in Kommentaren werden nicht ausgeführt!
  \end{itemize}
\end{frame}

\begin{frame}[fragile]{do-File Ic}
\begin{itemize}
  \item Darunter folgt noch
  \index{clear all}
  \index{set more off}
  
  \begin{lstlisting}
  ** Den Speicher leeren
  clear all
  ** more Funktion abschalten
  set more off
  \end{lstlisting}
  \end{itemize}
      
   \begin{tikzpicture}[transform shape, rotate=5, overlay]
\node at (8.5,-.5) [mybox] (box) {%
    \begin{minipage}[t!]{0.35\textwidth}
    \tiny\textcolor{black}{\texttt{clear all löscht alles aus dem Speicher, damit ist sichergestellt, dass alles was ausgeführt wird, auf das/die ausgeführte/n do-File/s zurück geht.}}
    \end{minipage}
    };
\end{tikzpicture}

\end{frame}


\begin{frame}[fragile]{do-File II}
 \begin{itemize}
  \item Testdaten eingelesen. \index{Laden!use} \index{Laden}

  \begin{lstlisting}
  use "D:/Daten/testdaten.dta"
  \end{lstlisting}

  \item Beschreiben \index{Beschreiben!describe}
  
  \begin{lstlisting}
  describe
  \end{lstlisting}
 
  \item Hilfe zum Befehl \index{Hilfe!help}
  \begin{lstlisting}
  help describe
  \end{lstlisting}

 \end{itemize}
\end{frame}

\subsection{help}
\begin{frame}{help} \index{Hilfe}\index{Hilfe!help}
\begin{scriptsize}
  \texttt{\underline{tab}ulate \textcolor{Statakeywords}{varname1} \textcolor{Statakeywords}{varname2} [\textcolor{Statakeywords}{if}] [\textcolor{Statakeywords}{in}] [\textcolor{Statakeywords}{weight}] [, options]} \\ \vspace{.5cm}

  \begin{tabular}{ll}
  options & Description \\
  \midrule
  Main & \\
  chi2 &               report Pearson's chi-squared \\
  exact[(\#)] &         report Fisher's exact test \\
  \dots & \\
  \end{tabular} \vspace{.5cm}

  \texttt{\underline{reg}ress \textcolor{Statakeywords}{depvar} [\textcolor{Statakeywords}{indepvars}] [\textcolor{Statakeywords}{if}] [\textcolor{Statakeywords}{in}] [\textcolor{Statakeywords}{weight}] [, options]} \\
  \vspace{.5cm}
  
  \begin{tabular}{ll}
  options & Description \\
  \midrule
  Model & \\
  noconstant & suppress constant term \\
  hascons & has user-supplied constant \\
  \dots & \\
  \end{tabular}

\end{scriptsize}
\end{frame}

\subsection{findit}
\begin{frame}[fragile]{findit} \index{Suchen!findit}
Und wie kommt man an Befehle?
\begin{lstlisting}
  findit describe
\end{lstlisting}

   \begin{tikzpicture}[transform shape, rotate=5, overlay]
\node at (8.5,-.5) [mybox] (box) {%
    \begin{minipage}[t!]{0.35\textwidth}
    \tiny\textcolor{black}{\texttt{findit kann Hinweise liefern, muss aber nicht. Suchen Sie in jedem Fall nach den englischen Fachbegriffen.}}
    \end{minipage}
    };
\end{tikzpicture}

\end{frame}


\begin{frame}[fragile]{do-File III}
 Verschiedene Aufgaben, verschiedene do-Files. \index{do-Files}
 \begin{itemize}
  \item Daten einlesen
  \item Aufbereitung
  \item Deskriptive Auswertungen
  \item usw.
 \end{itemize}
 
    \begin{tikzpicture}[transform shape, rotate=5, overlay]
\node at (8.5,-.5) [mybox] (box) {%
    \begin{minipage}[t!]{0.35\textwidth}
    \tiny\textcolor{black}{\texttt{Nicht jedes do-File wird immer gebraucht. Lesen Sie z.~B. einmal die Daten ein, bereiten Sie sie einmal auf. Das muss nicht für jede neue Kreuztabelle erfolgen. Zeit ist Geld!}}
    \end{minipage}
    };
\end{tikzpicture}
\end{frame}


\begin{frame}[fragile]{do-File IV}

 do-Files können ihrerseits auch aus do-Files aufgerufen werden \index{do-Files} \index{do}
\begin{lstlisting}
  ** do-Files aufrufen
  do DatenEinlesen.do
  do Aufbereitung.do
  do DeskriptiveAuswertungen.do
\end{lstlisting}
\begin{tikzpicture}[transform shape, rotate=10, overlay]
\node at (7.5,-1.5) [mybox] (box) {%
    \begin{minipage}[t!]{0.35\textwidth}
    \tiny\textcolor{black}{\texttt{Dadurch wird die Syntax schlanker und die do-Files werden garantiert in der richtigen Reihenfolge aufgerufen.}}
    \end{minipage}
    };
\end{tikzpicture}

\end{frame}



\section{Arbeiten}
\subsection{Übersicht}
\begin{frame}[fragile]{Übersicht}
Übersicht der eingelesene Daten \index{Beschreiben} \index{Beschreiben!describe} \index{Beschreiben!codebook} \index{list} \index{browse} \index{Hilfe!help}

  \begin{lstlisting}
  describe
  codebook
  list
  browse
  \end{lstlisting}
  
Nach \texttt{help describe} kennt man
  \begin{lstlisting}
  describe, simple
  \end{lstlisting}
  
  \begin{tikzpicture}[transform shape, rotate=10, overlay]
\node at (8.5,-.5) [mybox] (box) {%
    \begin{minipage}[t!]{0.25\textwidth}
    \tiny\textcolor{black}{\texttt{In der Hilfe steht, ob einem Befehl noch einzelne Variablen oder Variablenlisten übergeben werden können.}}
    \end{minipage}
    };
\end{tikzpicture}

\end{frame}

\subsection{Deskriptiv}
\begin{frame}[fragile]{Deskriptive Auswertungen}
Deskriptive Auswertungen von Variablen\footnote{\texttt{var1} und \texttt{var2} sind hier Platzhalter.}

\index{Tabelle!tabulate} \index{Beschreiben!summarize} \index{Hilfe!help}
  \begin{lstlisting}
  tabulate var1
  tabulate var2
  tabulate var1 var2
  summarize var1
  summarize var1-var4
  \end{lstlisting}

Nach \texttt{help summarize} kennt man
  \begin{lstlisting}
  summarize var1, detail
  \end{lstlisting}

\end{frame}