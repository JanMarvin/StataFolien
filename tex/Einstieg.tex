%%%
%%% Introduction
%%% 
\part{Introduction}
\begin{frame}
\thispagestyle{empty}
\textbf{\huge{Introduction}}
\end{frame}

\begin{frame}{Introduction Contents}
 \tableofcontents
\end{frame}


\section{Datatypes}
\subsection{Syntaxfiles}
\begin{frame}[fragile]{Syntax}
 We write Syntax.
 \begin{itemize}
  \item Reproducable
  \item Less \dots
  \begin{itemize}
   \item clicking
   \item repeating
   \item error-prone
  \end{itemize}
 \end{itemize}
\end{frame}

\begin{frame}[fragile]{Syntaxfiles}
  \begin{itemize}
  \item Stata syntax is handled in do-files. \index{do-file}
  \item Starting the editor \index{do-file!doedit}
  \begin{lstlisting}
  doedit
  \end{lstlisting}
  \item Fileending of do-files is .do.
  \item do-files can be opend and modified with any texteditor.
  \end{itemize}
\end{frame}

\subsection{Datafiles}
\begin{frame}{secondary data}
 \begin{itemize}
  \item primary data
  \item secondary data
 \end{itemize}

 \begin{tikzpicture}[transform shape, rotate=10, overlay]
\node at (7.5,-1.5) [mybox] (box) {%
    \begin{minipage}[t!]{0.35\textwidth}
    \tiny\textcolor{black}{\texttt{We use data. We do not gather data. We will not create or edit data.}}
    \end{minipage}
    };
\end{tikzpicture}

\end{frame}

\begin{frame}{dta-files}
\index{dta-file}
  \begin{itemize}
  \item Data is stored in dta-files.
  \item Data filending is .dta.
  \item Stata dataobjects are readable only by Stata and some other statistical software e.g. R (\cite{R13}).
  \end{itemize}
\end{frame}

\begin{frame}[fragile]{Importing data}
  \begin{itemize}
    \item Showing the content of a directory \index{directory!dir}
    \begin{lstlisting}
    dir
    \end{lstlisting}
    \item Changing the directory \index{directory!cd}
    \begin{lstlisting}
    cd "D:/Data/"
    \end{lstlisting}
    \item Importing data \index{use} \index{Import}
    \begin{lstlisting}
    use <dataname.dta>
    \end{lstlisting}
  \end{itemize}
  \index{Umlauts}
  \index{Whitespaces}
  
   \begin{tikzpicture}[transform shape, rotate=10, overlay]
\node at (7.5,-0.5) [mybox] (box) {%
    \begin{minipage}[t!]{0.35\textwidth}
    \tiny\textcolor{black}{\texttt{Avoid directory- or filenames containing umlauts or whitespaces. Stata does not like those.}}
    \end{minipage}
    };
\end{tikzpicture}
  
\end{frame}

\section{do-files}
\begin{frame}[fragile]{do-file I}
  \begin{itemize}
  \item file header: should have informations like filename, author, creation and/or modification date, a short description of its content and for longer files a short table of contents.
  \index{file header}
  \index{do-file}
  \index{comment}
    
  \begin{lstlisting}
  /*
  Name: read-soep.do
  Author: Garbuszus
  Date: 2016-02-28
  Content: Importing SOEP-data
  */
  \end{lstlisting}
  \end{itemize}
\end{frame}

\begin{frame}{do-file Ia}
Document your code! \index{comment}
\begin{itemize}
\item Teamwork
\item Obliviousness 
\end{itemize}

  
     \begin{tikzpicture}[transform shape, rotate=10, overlay]
\node at (7.5,-0.5) [mybox] (box) {%
    \begin{minipage}[t!]{0.35\textwidth}
    \tiny\textcolor{black}{\texttt{Nobody likes to study your syntax, document it. Document a lot and thoroughly, you will forget!}}
    \end{minipage}
    };
\end{tikzpicture}

\end{frame}


\begin{frame}[fragile]{do-file Ib}
\begin{itemize} \index{comment!commenting}
   \item \texttt{/*} (begin) und \texttt{*/} (end) define a comment. Alternative this can be done at the begin of a line with \texttt{*} or inline with \texttt{//}
  
  \begin{lstlisting}
  ** We write code for Stata version 8 and later
  version 8 // This runs with Stata 8 and might not work with Stata 7
  \end{lstlisting}
  \item Commands in comments will be skipped!
  \end{itemize}
\end{frame}

\begin{frame}[fragile]{do-file Ic}
\begin{itemize}
  \item This is followed by
  \index{clear all}
  \index{set more off}
  
  \begin{lstlisting}
  ** remove everything from our data memory
  clear all
  ** disables more functionalty
  set more off
  \end{lstlisting}
  \end{itemize}
      
   \begin{tikzpicture}[transform shape, rotate=5, overlay]
\node at (8.5,-.5) [mybox] (box) {%
    \begin{minipage}[t!]{0.35\textwidth}
    \tiny\textcolor{black}{\texttt{clear all errases everything from Statas memory, securing, that everything in the following do files is run on exactly the data we define and not on some earlyer created data artifacts. It's reproducable!}}
    \end{minipage}
    };
\end{tikzpicture}

\end{frame}


\begin{frame}[fragile]{do-file II}
 \begin{itemize}
  \item Import some testdata \index{Import!use} \index{Import}

  \begin{lstlisting}
  use "D:/Data/testdaten.dta"
  \end{lstlisting}

  \item Describe \index{Describe!describe}
  
  \begin{lstlisting}
  describe
  \end{lstlisting}
 
  \item Help for command \index{Help!help}
  \begin{lstlisting}
  help describe
  \end{lstlisting}

 \end{itemize}
\end{frame}

\subsection{help}
\begin{frame}{help} \index{Help}\index{Help!help}
\begin{scriptsize}
  \texttt{\underline{tab}ulate \textcolor{Statakeywords}{varname1} \textcolor{Statakeywords}{varname2} [\textcolor{Statakeywords}{if}] [\textcolor{Statakeywords}{in}] [\textcolor{Statakeywords}{weight}] [, options]} \\ \vspace{.5cm}

  \begin{tabular}{ll}
  options & Description \\
  \midrule
  Main & \\
  chi2 & report Pearson's chi-squared \\
  exact[(\#)] & report Fisher's exact test \\
  \dots & \\
  \end{tabular} \vspace{.5cm}

  \texttt{\underline{reg}ress \textcolor{Statakeywords}{depvar} [\textcolor{Statakeywords}{indepvars}] [\textcolor{Statakeywords}{if}] [\textcolor{Statakeywords}{in}] [\textcolor{Statakeywords}{weight}] [, options]} \\
  \vspace{.5cm}
  
  \begin{tabular}{ll}
  options & Description \\
  \midrule
  Model & \\
  noconstant & suppress constant term \\
  hascons & has user-supplied constant \\
  \dots & \\
  \end{tabular}

\end{scriptsize}
\end{frame}

\subsection{findit}
\begin{frame}[fragile]{findit} \index{Search!findit}\index{findit}
And how does one find commands?
\begin{lstlisting}
  findit describe
\end{lstlisting}

   \begin{tikzpicture}[transform shape, rotate=5, overlay]
\node at (8.5,-.5) [mybox] (box) {%
    \begin{minipage}[t!]{0.35\textwidth}
    \tiny\textcolor{black}{\texttt{findit can be helpfull, but it's not guaranteed.}}
    \end{minipage}
    };
\end{tikzpicture}

\end{frame}


\begin{frame}[fragile]{do-File III}
 Different tasks, different do-files. \index{do-file}
 \begin{itemize}
  \item Import data
  \item Preparation
  \item Descriptive analysis
  \item etc.
 \end{itemize}
 
    \begin{tikzpicture}[transform shape, rotate=5, overlay]
\node at (8.5,-.5) [mybox] (box) {%
    \begin{minipage}[t!]{0.35\textwidth}
    \tiny\textcolor{black}{\texttt{Not every do-file is required to be rerun all the time. For instance importing and preparation of the data should be a one time task. It's not required to do this for every single crosstable or boxplot. Time is money!}}
    \end{minipage}
    };
\end{tikzpicture}
\end{frame}


\begin{frame}[fragile]{do-file IV}

 since do-files may be executed by other do-files nesting some of them in a master file helps with housekeeping \index{do-file} \index{do}
\begin{lstlisting}
  ** call other do-files
  do "01-Data_Import.do"
  do "02-Data_Preparation.do"
  do "03-Descriptive_Analysis.do"
\end{lstlisting}
\begin{tikzpicture}[transform shape, rotate=10, overlay]
\node at (7.5,-1.5) [mybox] (box) {%
    \begin{minipage}[t!]{0.35\textwidth}
    \tiny\textcolor{black}{\texttt{This tightens your syntax and the correct order of do-files is asured.}}
    \end{minipage}
    };
\end{tikzpicture}

\end{frame}



\section{Working}
\subsection{Description of data in memory}
\begin{frame}[fragile]{Description of data in memory}
Description of data in memory \index{Describe} \index{Describe!describe} \index{Describe!codebook} \index{list} \index{browse} \index{Help!help}

  \begin{lstlisting}
  describe
  codebook
  list
  browse
  \end{lstlisting}
  
Studying \texttt{help describe} shows you
  \begin{lstlisting}
  describe, simple
  \end{lstlisting}
  
  \begin{tikzpicture}[transform shape, rotate=10, overlay]
\node at (8.5,-.5) [mybox] (box) {%
    \begin{minipage}[t!]{0.25\textwidth}
    \tiny\textcolor{black}{\texttt{Help files contain information whether or not a command requires a single variable or a variablelist.}}
    \end{minipage}
    };
\end{tikzpicture}

\end{frame}

\subsection{Descriptive}
\begin{frame}[fragile]{Descriptive Analysis}
Descriptive analysis of variables \footnote{\texttt{var1} and \texttt{var2} are placeholders.}

\index{Table!tabulate} \index{Describe!summarize} \index{Help!help}
  \begin{lstlisting}
  tabulate var1
  tabulate var2
  tabulate var1 var2
  summarize var1
  summarize var1-var4
  \end{lstlisting}

Studying \texttt{help summarize} shows you
  \begin{lstlisting}
  summarize var1, detail
  \end{lstlisting}

\end{frame}