%%%
%%% Statistik
%%%
\part{Statistik}
\begin{frame}
\thispagestyle{empty}
\textbf{\huge{Statistik}}
\end{frame}

\begin{frame}{Statistik Inhalt}
 \tableofcontents
\end{frame}


\section{Maßzahlen}
\begin{frame}[fragile]{Maßzahlen I}
Arithmetisches Mittel und ein paar Maßzahlen \index{Mittelwert!mean} \index{Beschreiben!summarize} \index{Beschreiben!detail} \index{Mittelwert} \index{Maßzahlen}
\begin{lstlisting}
  mean sex
  summarize sex
  summarize sex, detail
\end{lstlisting}

  \begin{tikzpicture}[transform shape, rotate=10, overlay]
\node at (8,-1.5) [mybox] (box) {%
    \begin{minipage}[t!]{0.35\textwidth}
    \tiny\textcolor{black}{\texttt{mean gibt den Standardfehler aus: $s/\sqrt{n}$. summarize gibt die Standardabweichung aus: $\sqrt{\sum(x-\bar{x})^2)/n}$.}}
    \end{minipage}
    };
\end{tikzpicture}
\end{frame}


\begin{frame}[fragile]{Maßzahlen II}
Minimum, Maximum, arithmetisches Mittel, Median, Anzahl und Quartile. \index{Minimum} \index{Minimum!min} \index{Maximum} \index{Maximum!max} \index{Mittelwert!mean} \index{Mittelwert} \index{Mittelwert!Arithmetisches Mittel} \index{Mittelwert!Median} \index{Quantile} \index{tabstat} \index{Range} \index{Range!range} \index{Quantile!q}
\begin{lstlisting}
  tabstat age, statistic(min max mean median p50)
  tabstat age, statistic(min max range mean count q) by(sex)
\end{lstlisting}

  \begin{tikzpicture}[transform shape, rotate=10, overlay]
\node at (8,-1.5) [mybox] (box) {%
    \begin{minipage}[t!]{0.35\textwidth}
    \tiny\textcolor{black}{\texttt{range = max - min}}
    \end{minipage}
    };
\end{tikzpicture}
\end{frame}


\begin{frame}[fragile]{Maßzahlen III}
Standardabweichung, Standardfehler, Varianz und Interquartilsabstand. \index{Standardabweichung} \index{Standardfehler} \index{Varianz} \index{Interquartilsabstand} \index{tabstat} \index{Standardabweichung!sd} \index{Varianz!var} \index{Standardfehler!sem} \index{Interquartilsabstand!iqr} \index{Quantile!q} \index{Schiefe} \index{Wölbung} \index{Schiefe!skewness} \index{Wölbung!kurtosis}
\begin{lstlisting}
 tabstat age, statistic(sd sem var q iqr)
\end{lstlisting}
Schiefe und Wölbung
\begin{lstlisting}
  tabstat age, statistics(skewness kurtosis)
\end{lstlisting}

  \begin{tikzpicture}[transform shape, rotate=10, overlay]
\node at (8,-1.5) [mybox] (box) {%
    \begin{minipage}[t!]{0.35\textwidth}
    \tiny\textcolor{black}{\texttt{iqr = q3 - q1}}
    \end{minipage}
    };
\end{tikzpicture}
\end{frame}

\section{Tabellen}
\begin{frame}[fragile]{Tabellen} \index{Tabelle!tab} \index{Tabelle!tab, summarize()} \index{Tabelle}
Wir hatten bereits \texttt{summarize} und \texttt{tab}, jetzt kombinieren wir
\begin{lstlisting}
  tab sex, summarize(age)
\end{lstlisting}
\end{frame}

\subsection{Kreuztabellen}
\begin{frame}[fragile]{Kreuztabellen} \index{Tabelle!tab} \index{Tabelle!tab1} \index{Tabelle!tab2} \index{Tabelle!Kreuztabelle}
Kreuztabellen erzeugen mit
\begin{lstlisting}
  tab sex east
\end{lstlisting}
Weitere Tabellen
\begin{lstlisting}
  ** Tabellen von jeder der drei Variablen
  tab1 sex age edu
  ** Kreuztabellen ab bc ac
  tab2 sex age edu
\end{lstlisting}
\end{frame}

\subsection{Chi, V und Phi}
\begin{frame}[fragile]{Kreuztabellen II} \index{Tabelle!Kreuztabelle} \index{Tabelle!$\chi^2$} \index{Tabelle!chi-square} \index{Tabelle!chi} \index{Tabelle!tab} \index{Tabelle!Cramers V} \index{Tabelle!tab exp} \index{Tabelle!tab col} \index{Tabelle!tab row}
Die Statistik I Vorlesung rekapitulieren, wir machen ein paar Tests \footnote{Oder nachschlagen z.~B. bei \textcite{Krebs10} oder \textcite{Agresti09}. }
\begin{lstlisting}
  ** chi-quadrat
  tab sex east, chi
  ** cramers v
  tab sex east, V
  ** wer per Hand nachrechnen will
  tab sex east, exp col row
  ** phi manuell ausrechnen
  di (934*426-441*1026) / (1375*1452*1960*867)^(1/2)
  di 2827 *(-.02963311)^2 // == chi^2
\end{lstlisting}

\end{frame}

\begin{frame}[fragile]{Numlabel}
Falls man die numerischen Werte auch im Label haben möchte \index{Label!numlabel}
\begin{lstlisting}
  ** numerische Label an für alle Variablen
  numlabel _all, add
  ** numerische Label aus
  numlabel _all, remove
\end{lstlisting}

\end{frame}

\section{Inferenz}
\begin{frame}[fragile]{T-Test} \index{T-Test!ttest} \index{T-Test}
Teststatistik
\begin{lstlisting}
  ** Vergleich Männer Frauen in West Ost
  tab sex, sum(east)
  ttest sex, by(east)
\end{lstlisting}
\end{frame}

\begin{frame}[fragile]{Und noch viel mehr \dots} \index{Regression!regress} \index{Regression!logit} \index{Regression!mlog}
Regression
\begin{lstlisting}
  regress
  logit
  mlog
\end{lstlisting}
\end{frame}

\begin{frame}
\thispagestyle{empty}
 Fortsetzung folgt.
\end{frame}
